\documentclass[12pt,a4paper]{article}

% ----------------------
% Pacotes essenciais
% ----------------------
\usepackage[utf8]{inputenc}
\usepackage[T1]{fontenc}
\usepackage[brazil]{babel}
\usepackage{amsmath, amssymb}
\usepackage{physics}
\usepackage{graphicx}
\usepackage{float}
\usepackage{booktabs}
\usepackage{hyperref}
\usepackage{geometry}
\usepackage{caption}
\usepackage{subcaption}

% ----------------------
% Margens
% ----------------------
\geometry{
  left=3cm,
  right=2.5cm,
  top=3cm,
  bottom=2.5cm
}

% ----------------------
% Informações do documento
% ----------------------
\title{
\Large \textbf{Dinâmica Não Linear e Emergência de Computação em Sistemas Simulados}\\
\vspace{0.3cm}
\normalsize Um estudo baseado na Rede de Osciladores de Kuramoto.
}

\author{
Anônimo\\
\vspace{0.2cm}
\small Projeto independente — 2026
}

\date{\today}

% ----------------------
\begin{document}
\maketitle

% ----------------------
\begin{abstract}

\end{abstract}

\section{Introdução}
Quando um sistema em questão varia seu estado com o passar do tempo ou oscila periodicamente, toda e qualquer análise que se possa efetuar sobre a conjuntura faz parte do campo da \textit{\textbf{dinâmica}}.


\section{Modelo Matemático}




\section{Observáveis e Métricas}


\section{Resultados}

\subsection{Sistema não estimulado}



\subsection{Sistema sob estímulo externo}


\section{Discussão}


\section{Conclusão}


\section*{Trabalhos Futuros}

Investigações futuras incluirão a introdução de topologias de rede mais realistas e estímulos derivados de fluxos reais de dados.

\begin{thebibliography}{9}

\bibitem{strogatz}
S. H. Strogatz,
\textit{Nonlinear Dynamics and Chaos},
Westview Press, 2015.

\bibitem{sync}
S. H. Strogatz,
\textit{Sync},
Hyperion, 2003.

\end{thebibliography}

\end{document}
